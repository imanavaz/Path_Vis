%%%% Proceedings format for most of ACM conferences (with the exceptions listed below) and all ICPS volumes.
\documentclass[sigconf]{acmart}
%%%% As of March 2017, [siggraph] is no longer used. Please use sigconf (above) for SIGGRAPH conferences.

%%%% Proceedings format for SIGPLAN conferences 
% \documentclass[sigplan, anonymous, review]{acmart}

%%%% Proceedings format for SIGCHI conferences
% \documentclass[sigchi, review]{acmart}

%%%% To use the SIGCHI extended abstract template, please visit
% https://www.overleaf.com/read/zzzfqvkmrfzn

\settopmatter{printacmref=false}

\usepackage{booktabs} % For formal tables
\usepackage{graphicx}


% Copyright
%\setcopyright{none}
%\setcopyright{acmcopyright}
%\setcopyright{acmlicensed}
\setcopyright{rightsretained}
%\setcopyright{usgov}
%\setcopyright{usgovmixed}
%\setcopyright{cagov}
%\setcopyright{cagovmixed}


% DOI
\acmDOI{10.475/123_4}

% ISBN
\acmISBN{123-4567-24-567/08/06}

%Conference
\acmConference[ACMMM2017]{ACM Multimedia}{October 2017}{Mountain View, CA USA} 
\acmYear{2017}
\copyrightyear{2017}

\acmPrice{15.00}


\begin{document}
\title{Interactive Visualisation for Travel Route Recommendations}

\author{Dawei Chen*$\ddagger$, Dongwoo Kim*, Lexing Xie*$\ddagger$,  Minjeong Shin*, Aditya Menon*$\ddagger$, Cheng Soon Ong*$\ddagger$, Iman Avazpour$\dagger$, John Grundy$\dagger$}
\affiliation{%
  \institution{*The Australian National University, $\ddagger$Data61, $\dagger$Deakin University}
%  \streetaddress{P.O. Box 1212}
%  \city{Dublin} 
%  \state{Ohio} 
%  \postcode{43017-6221}
}
\email{{u5708856,dongwoo.kim,lexing.xie,minjeong.shin,aditya.menon, chengsoon.ong}@anu.edu.au, {iman.avazpour,j.grundy}@deakin.edu.au}


%\authornote{The secretary disavows any knowledge of this author's actions.}
%\affiliation{%
%  \institution{Institute for Clarity in Documentation}
%  \streetaddress{P.O. Box 1212}
%  \city{Dublin} 
%  \state{Ohio} 
%  \postcode{43017-6221}
%}
%\email{webmaster@marysville-ohio.com}
%
%\author{Lars Th{\o}rv{\"a}ld}
%\affiliation{%
%  \institution{The Th{\o}rv{\"a}ld Group}
%  \streetaddress{1 Th{\o}rv{\"a}ld Circle}
%  \city{Hekla} 
%  \country{Iceland}}
%\email{larst@affiliation.org}

% The default list of authors is too long for headers}
\renewcommand{\shortauthors}{D. Chen et al.}


\begin{abstract}
We present an interactive visualisation tool for recommending travel trajectories. 
This system is based on new machine learning formulations for the trajectory recommendation problem~\cite{chen2016learning,chen2017SR}. It is designed to explain structured predictions made by several new algorithms, and to assist comparative visual analysis of different candidate trajectories. 
One challenge for visualizing the properties of a structured object such as a trajectory is to present and compare information from a large number of modalities. We tackle this challenge by breaking down contributions from different geographical and user behavior features, and also separating contributions from individual points-of-interest, and those involving two consecutive points on a route. The system also supports detailed quantitative interrogation by comparing a large number of features for two or more points. 
Trajectory visualisation can potentially benefit any user of online maps by increasing transparency on machine-learning algorithms and assist decision-making.  More broadly, the design of this system can inform visualisations of other structured prediction tasks, such as for sequences or trees. 
% While routing and location recommendation has been extensively explored in research and commercially available systems, 
%
%We propose a novel travel route visualisation tool to help an interaction between tourists and route recommendation system. 
%While the route recommendation algorithm shows promising results in a laboratory setup on benchmark dataset, the process of recommendation is still invisible to end-users who would benefit the information used to recommend the routes. 
%Based on a structured prediction algorithm tailored for the route recommendation, we propose a route visualisation which aims to reduce the gap between the end-users and recommendation system by visualising recommendation scores on various attributes of the suggested routes.
\end{abstract}

%
% The code below should be generated by the tool at
% http://dl.acm.org/ccs.cfm
% Please copy and paste the code instead of the example below. 
%
\begin{CCSXML}
<ccs2012>
<concept>
<concept_id>10002951.10003317.10003338.10003343</concept_id>
<concept_desc>Information systems~Learning to rank</concept_desc>
<concept_significance>500</concept_significance>
</concept>
<concept>
<concept_id>10003120.10003145</concept_id>
<concept_desc>Human-centered computing~Visualization</concept_desc>
<concept_significance>500</concept_significance>
</concept>
</ccs2012>
\end{CCSXML}

\ccsdesc[500]{Information systems~Learning to rank}
\ccsdesc[500]{Human-centered computing~Visualization}
% We no longer use \terms command
%\terms{Theory}

\keywords{Visualisation, Recommendation}

% Used in some conference proceedings e.g. sigplan and sigchi
 \begin{teaserfigure}
 \centering
   \includegraphics[width=0.8\textwidth]{figure/sample_map.png}
   \caption{Travel route visualisation system. Given a starting POI and a number of POI to be visited, the system recommends a set of routes from a history of previous tourists.}
   \label{fig:overview}
 \end{teaserfigure}

\maketitle


% !TEX root = ./acmmm2017.tex

\section{Introduction}
%background
Sequence ranking has emerged as an important tool for solving diverse problems such as travel route and music playlist recommendations. 
Unlike the classical ranking algorithm where items are considered independently, the sequence ranking algorithm requires modelling a structure between items and suggests a set of items as a whole. 
For example, consider recommending a trajectory of points of interest (POI) in a city to a visitor. 
While the classical ranking algorithm can learn a user's preference for each individual location, it may ignore the distances between them and could suggest a long trajectory, which should be shorter in optimal routing. 
Several sequence ranking algorithms have been proposed to solve this problem and demonstrated superior performance compared to classical ranking algorithms. 
Nonetheless, recommendation algorithms for sequences and trajectories~\cite{chen2016learning,chen2017SR} have many components and can be difficult for a user to understand. This is part of the general challenge of introducing transparency and accountability for machine learning algorithms~\cite{fatml}. 
%a remaining challenge is to construct an interactive recommendation system so that a user can analyse the suggested sequences and plan a better trip.

%approach
In this paper, we tackle the problem of sequence visualisation, in particular, for travel routes recommendation. 
We define a travel route as a sequence of POIs and formulate the sequence ranking problem as a structured prediction problem. 
Based on a diverse set of features for individual and pairs of POIs, we train the prediction model with trajectory data extracted from geo-tagged photos. 
To visualise the suggested routes, we develop a novel tool that efficiently displays multiple suggested routes, which helps users understand the process behind the recommendations.
Specifically, our system decomposes a total score of each route into a set of features and their corresponding scores and shows the total score as a stacked bar plot of the features.
The system also visualises the difference between POIs in a single route to show how POIs in that route can exhibit vast diversity. 
This visualisation helps tourists who want to have diverse experiences by choosing the best route among the set of recommendations. Generalising to routes in general, such a visualisation could also help users of online mapping apps to make decisions on recommended travel routes, such as trading off distance, traffic, and scenery. 
\section{Structured Prediction}
% don't need very details of the algorithm
% need problem definition
% copied from nips paper
Travel route recommendation problems involve a set of POIs in a city. 
Given a trajectory query $\mathbf{x} = (s, l)$, comprising a start POI $s$ and trip length $l$, the goal is to suggest one or more sequences of POIs that maximise some notion of utility.

We first cast travel recommendation as a structured prediction problem, which allows us to leverage the well-studied literature of structured SVMs (SSVM)~\cite{tsochantaridis2005large,joachims2009predicting}. 
There are two obstacles that prevent us from applying SSVM directly to the sequence recommendation problem; first, there would be multiple ground-truth routes among a set of POIs, second, a naive application of SSVM would generate a loop during prediction time. 
To incorporate multiple ground-truth routes in the learning phase, we take an idea from the ranking objective which prevents the ground-truth routes from competing with each others~\cite{rendle2009bpr}. 
To eliminate possible loops in prediction, we adopt the serial list Viterbi algorithm~\cite{seshadri1994list,nill1995list,nilsson2001sequentially}.
We finally trained our model on trajectory data extracted from Flickr photos taken in Melbourne~\cite{chen2016learning}.

From a visualisation perspective, an important advantage of the SSVM is the explicit representation of feature scores in its final decision process. Especially, in our case, we can disassemble the final score of a route into feature scores of each POI and each transition between two adjacency POIs. 
We use hand-crafted POI features such as the category, popularity, and average visit duration of previous tourists and also crafted transition features such as the distance and neighbourhood of two POIs to maximise the interpretability of the outcome.

\begin{figure}[t!]
\includegraphics[width=0.9\linewidth]{figure/sample_stack.png}
\caption{Visualisation of POI scores for top ten recommended routes. Each bar from left to right represent a relative score of each POI along the route, and the total length of stacked bars represents the total score of the suggested route.}
\label{fig:stack}
\end{figure}
\section{Visualisation}
%what are the features we want to emphasis?
%make a bullet list of something like that
Our goal is to design an interactive visualisation system on top of the structured prediction framework.
Figure~\ref{fig:overview} shows the overview of a live demo system, which consists of four major components: a map to display suggested routes, an input box for user query on the left side, a stacked score of routes on the upper right side, and a radar chart to compare multiple POIs on the lower right side. 
The role and the construction of three major components, except the main map, are as follow:

%\begin{itemize}
\textbf{Query input}: A query consists of a starting POI and a trip length. 
Users can choose the starting POI by clicking icons on the map and can adjust the slide to set the trip length. 
In addition, we support three different travelling modes: bicycling, walking, and driving.
Based on the different mode, we optimise suggested routes between POIs in a sequence.

\textbf{Route score visualisation}: Ranks of candidate routes are determined by the total scores from the SSVM. 
The score of each route can be decomposed into each POI and edge scores along the route. 
We adopt the LineUp framework~\cite{gratzl2013lineup} designed to support the visualisation of multi-attribute ranking via stacked representation. 
Figure~\ref{fig:stack} shows the stacked representation of the top ten scored routes, where the proportion of each bar indicates the importance of each POI in the route.

\textbf{POI feature visualisation}: We further provide a tool to analyse the variation between POIs in a single route. 
For example, in Figure~\ref{fig:radar}, we compare two POIs, the Queen Victoria Market and Melbourne Aquarium, in terms of POI features and their importance in the suggested route. 
%\end{itemize}

\begin{figure}[t!]
\includegraphics[width=0.8\linewidth]{figure/sample_radar.png}
\caption{Comparison between the Queen Victoria Market and Melbourne Aquarium. The Queen Victoria Market has a high score on \textit{shopping} and \textit{city precincts} features whereas the Melbourne aquarium has a high score on \textit{Entertainment} feature.}
\label{fig:radar}
\end{figure}
\section{Conclusion}
%summary
In this demonstration, we showcase an interactive route analyser which helps the interaction between users and route recommendation systems. 
The system benefits from the explicit feature construction of the structured prediction model and visualises recommended routes along with relevant information on both route and POI level.



\bibliographystyle{ACM-Reference-Format}
\bibliography{sigproc} 

\end{document}
